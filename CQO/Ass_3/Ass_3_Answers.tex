% Created 2022-12-12 Mon 18:58
% Intended LaTeX compiler: pdflatex
\documentclass[a4paper,11pt]{article}
\usepackage[utf8]{inputenc}
\usepackage[T1]{fontenc}
\usepackage{graphicx}
\usepackage{longtable}
\usepackage{wrapfig}
\usepackage{rotating}
\usepackage[normalem]{ulem}
\usepackage{amsmath}
\usepackage{amssymb}
\usepackage{capt-of}
\usepackage{hyperref}
\usepackage[margin=1in]{geometry}
\usepackage{titlesec}
\usepackage{caption}
\usepackage{subcaption}
\usepackage{lipsum}
\author{Varghese Reji}
\date{}
\title{Classical and Quantum Optics\\\medskip
\large Assignment-2 Answers}
\hypersetup{
 pdfauthor={Varghese Reji},
 pdftitle={Classical and Quantum Optics},
 pdfkeywords={},
 pdfsubject={},
 pdfcreator={Emacs 28.2 (Org mode 9.5.5)}, 
 pdflang={English}}
\begin{document}

\maketitle

\section*{Problem 1}
\label{sec:orged36dcd}

A beam with a photon flux of 1000 photons s\textsuperscript{-1} is incident on a detector with a quantum efficiency of 20\%. If the time interval of the counter is set to 10s, calculate the average and standard deviation of the photocount number for the followin g scenarios:
\begin{description}
\item[{(a)}] the light has Poissonian statistics;
\item[{(b)}] the light has super-Poissonian statistics with \(\Delta\) n=2\texttimes{} \(\Delta\) n\textsubscript{Poisson} ;
\item[{(c)}] the light is in a photon number state.
\end{description}

\subsection*{Answer}
\label{sec:orge5a9ebf}
\(\phi\) = 1000/s, \(\eta\)=20\%, t=10s

\begin{description}
\item[{(a)}] \(\bar{n} = \frac{L \phi}{c}\)

But, \(L=ct \Rightarrow \bar{n} = t \phi\)

Then, \(\bar{n} = 10000\).\$

\(\Delta n = \sqrt{n} = 100\)

The photocount number is given by, $$(\Delta N)^2 = \eta^2(\Delta n)^2 + \eta(1-\eta) \bar{n}$$

\(\Rightarrow\), $$\Delta N = 44.721$$.

$$\bar{N} = \eta \bar{n} = 2000$$
\end{description}


\begin{description}
\item[{(b)}] Given \(\Delta n = 2\times \Delta n_{Poisson}\).
i.e., \(\Delta n = 89.442\)
\end{description}



We know that, for super poissonian statistics,

$$(\Delta n)^2 = \bar{n} + \bar{n}^2$$.

\(\Rightarrow\)

$$ \bar{n} + \bar{n}^2 = 4 \bar{n}_{Poiss}^2 = 8000$$

By solving this,

$$\bar{n} =88.944$$


$$\Delta N = 18.28, \bar{N} = 17.79$$

\begin{description}
\item[{(c)}] In phton number state, \(\Delta n=0\). That means, there is no variation from mean value. Then, \(\bar{n} = 10000\)
\end{description}

\(\Rightarrow\)
$$\bar{N} = 2000, \Delta N = 40$$

\section*{Problem 2}
\label{sec:org1643581}
Calculate the values of g\textsuperscript{(2)}(0) for a monochromatic light wave with a square wave intensity modulation of \textpm{} 20\%.

\section*{Problem 3}
\label{sec:org0e78fa2}
The 632.8 nm line of a neon discharge lamp is Doppler-broadened with a linewidth of 1.5GHz. Sketch the second-order correlation function g\textsuperscript{(2)}(\(\tau\)) for \(\tau\) in range 0-1 ns.

\section*{Problem 4}
\label{sec:orgd33c5fb}
For the coherent states |\(\alpha\)> with \(\alpha\)=5, calculate
\begin{description}
\item[{(a)}] the mean photon number;
\item[{(b)}] the standard deviation in the photon number;
\item[{(c)}] the quantum uncertinity in the optical phase.
\end{description}

\section*{Problem 5}
\label{sec:org5c7c02c}
A ruby laser operating at 693 nm emits pulses of energy 1mJ. Calculate the quantum uncertinity in the phase of the laser light.

\section*{Problem 6}
\label{sec:orgda3dc16}
For the coherent state |\(\alpha\)> with \(\alpha\)=|\(\alpha\)|e\textsuperscript{i\(\phi\)}, show that \(<\alpha|\hat{X}_1|\alpha>=|\alpha|\cos\phi\) and \(<\alpha|\hat{X}_2|\alpha>=|\alpha|\sin\phi\). Show further that \(\Delta X_1 = \Delta X_2 = \frac{1}{2}\).

\section*{Problem 7}
\label{sec:orgab3a86f}
Prove that for two coherent states |\(\alpha\)> and |\(\beta\)>,
$$|<\alpha|\beta>|^2=\exp(-|\alpha-\beta|^2)$$
\end{document}